\documentclass[11pt,twoside,a4paper]{article}
\usepackage[utf8]{inputenc}
\usepackage{amsmath}
\usepackage{amsfonts}		
\usepackage{amssymb}
\usepackage{graphicx}
\usepackage{mathrsfs}
\usepackage{multicol}
\usepackage{tikz}
\usepackage{amsthm}
\usepackage{natbib}
\usetikzlibrary{arrows}
\usepackage{verbatim}
\usepackage[margin=2cm]{geometry}
\newcommand*\diff{\mathop{}\!\mathrm{d}}
\author{Eleanor McMurtry\\760505}
\title{Building a magneto-optical trap for rubidium atoms: a practical guide for the intrepid third year}
\begin{document}
\maketitle
\section*{Abstract}
Our goal for this project was to develop a procedure by which third-year undergraduate students could
build, test, and experiment with a magneto-optical trap (MOT). We successfully constructed a MOT and
demonstrated its effectiveness, though our setup required the use of expensive and fiddly fibre optic couples.
These were chosen for flexibility; an implementation of the experiment would need only minor modifications
to avoid using these.

This document is intended to be both a reflection of our experiences and a guide for a prospective
third year laboratory experiment. I will first outline some of the arelevant theory for the experiment,
then discuss the experimental procedures required to build the MOT;\@ I will conclude the document with a brief
discussion of our results.

\begin{center}    
\subsection*{Supervised by Andy McCulloch and Robert Scholten}
University of Melbourne School of Physics\\
Thanks also to Alexander Wood for his tireless assistance in the lab.
\end{center}
\pagebreak
\tableofcontents
\vfill
\pagebreak
\section{Introduction}
Optical trapping methodologies have their origin in the 1970s, when~\cite{ashkin} developed what has become known
as optical tweezers using electromagnetic gradients.
\section{Background theory}
\subsection{Hyperfine structure of rubidium-85}
\subsection{Non-allowed and dark transitions}
\subsection{Saturated absorption spectroscopy}
\subsection{Doppler cooling}
\subsection{The Zeeman effect}
\subsection{Repumping}
\section{Experimental procedure}
\subsection{Absorption spectroscopy}
\subsection{Saturated absorption spectroscopy}
\subsection{Vacuum cell setup}
\subsection{Rubidium sources and magnetic coils}
\section{Results and discussion}
\section{References}
\bibliography{papers}
\bibliographystyle{apa}
\end{document}